% Options for packages loaded elsewhere
\PassOptionsToPackage{unicode}{hyperref}
\PassOptionsToPackage{hyphens}{url}
%
\documentclass[
]{article}
\usepackage{amsmath,amssymb}
\usepackage{iftex}
\ifPDFTeX
  \usepackage[T1]{fontenc}
  \usepackage[utf8]{inputenc}
  \usepackage{textcomp} % provide euro and other symbols
\else % if luatex or xetex
  \usepackage{unicode-math} % this also loads fontspec
  \defaultfontfeatures{Scale=MatchLowercase}
  \defaultfontfeatures[\rmfamily]{Ligatures=TeX,Scale=1}
\fi
\usepackage{lmodern}
\ifPDFTeX\else
  % xetex/luatex font selection
\fi
% Use upquote if available, for straight quotes in verbatim environments
\IfFileExists{upquote.sty}{\usepackage{upquote}}{}
\IfFileExists{microtype.sty}{% use microtype if available
  \usepackage[]{microtype}
  \UseMicrotypeSet[protrusion]{basicmath} % disable protrusion for tt fonts
}{}
\makeatletter
\@ifundefined{KOMAClassName}{% if non-KOMA class
  \IfFileExists{parskip.sty}{%
    \usepackage{parskip}
  }{% else
    \setlength{\parindent}{0pt}
    \setlength{\parskip}{6pt plus 2pt minus 1pt}}
}{% if KOMA class
  \KOMAoptions{parskip=half}}
\makeatother
\usepackage{xcolor}
\usepackage[margin=1in]{geometry}
\usepackage{graphicx}
\makeatletter
\def\maxwidth{\ifdim\Gin@nat@width>\linewidth\linewidth\else\Gin@nat@width\fi}
\def\maxheight{\ifdim\Gin@nat@height>\textheight\textheight\else\Gin@nat@height\fi}
\makeatother
% Scale images if necessary, so that they will not overflow the page
% margins by default, and it is still possible to overwrite the defaults
% using explicit options in \includegraphics[width, height, ...]{}
\setkeys{Gin}{width=\maxwidth,height=\maxheight,keepaspectratio}
% Set default figure placement to htbp
\makeatletter
\def\fps@figure{htbp}
\makeatother
\setlength{\emergencystretch}{3em} % prevent overfull lines
\providecommand{\tightlist}{%
  \setlength{\itemsep}{0pt}\setlength{\parskip}{0pt}}
\setcounter{secnumdepth}{5}
\newlength{\cslhangindent}
\setlength{\cslhangindent}{1.5em}
\newlength{\csllabelwidth}
\setlength{\csllabelwidth}{3em}
\newlength{\cslentryspacingunit} % times entry-spacing
\setlength{\cslentryspacingunit}{\parskip}
\newenvironment{CSLReferences}[2] % #1 hanging-ident, #2 entry spacing
 {% don't indent paragraphs
  \setlength{\parindent}{0pt}
  % turn on hanging indent if param 1 is 1
  \ifodd #1
  \let\oldpar\par
  \def\par{\hangindent=\cslhangindent\oldpar}
  \fi
  % set entry spacing
  \setlength{\parskip}{#2\cslentryspacingunit}
 }%
 {}
\usepackage{calc}
\newcommand{\CSLBlock}[1]{#1\hfill\break}
\newcommand{\CSLLeftMargin}[1]{\parbox[t]{\csllabelwidth}{#1}}
\newcommand{\CSLRightInline}[1]{\parbox[t]{\linewidth - \csllabelwidth}{#1}\break}
\newcommand{\CSLIndent}[1]{\hspace{\cslhangindent}#1}
\usepackage{setspace}\doublespacing
\usepackage[fontsize=12pt]{scrextend}
\usepackage[left]{lineno}
\usepackage[labelformat=empty]{caption}
\usepackage{longtable}
\usepackage{booktabs}
\usepackage{booktabs}
\usepackage{longtable}
\usepackage{array}
\usepackage{multirow}
\usepackage{wrapfig}
\usepackage{float}
\usepackage{colortbl}
\usepackage{pdflscape}
\usepackage{tabu}
\usepackage{threeparttable}
\usepackage{threeparttablex}
\usepackage[normalem]{ulem}
\usepackage{makecell}
\usepackage{xcolor}
\ifLuaTeX
  \usepackage{selnolig}  % disable illegal ligatures
\fi
\IfFileExists{bookmark.sty}{\usepackage{bookmark}}{\usepackage{hyperref}}
\IfFileExists{xurl.sty}{\usepackage{xurl}}{} % add URL line breaks if available
\urlstyle{same}
\hypersetup{
  hidelinks,
  pdfcreator={LaTeX via pandoc}}

\author{}
\date{\vspace{-2.5em}}

\begin{document}

\textbf{S1 Text: Sensory processing in humans and mice fluctuates between
external and internal modes}\\
\strut \\

\textbf{Authors}:

Veith Weilnhammer\(^{1,2, 3}\), Heiner Stuke\(^{1,2}\), Kai
Standvoss\(^{1}\), Philipp Sterzer\(^{4}\)\\
\strut \\
\textbf{Affiliations}:

\(^{1}\) Department of Psychiatry, Charité-Universitätsmedizin Berlin,
corporate member of Freie Universität Berlin and Humboldt-Universität zu
Berlin, 10117 Berlin, Germany\\
\(^{2}\) Berlin Institute of Health, Charité-Universitätsmedizin Berlin
and Max Delbrück Center, 10178 Berlin, Germany\\
\(^{3}\) Helen Wills Neuroscience Institute, University of California
Berkeley, USA\\
\(^{4}\) Department of Psychiatry (UPK), University of Basel,
Switzerland\\
\strut \\

\textbf{Corresponding Author}:

Veith Weilnhammer, Helen Wills Neuroscience Institute, University of
California Berkeley, USA, email:
\href{mailto:veith.weilnhammer@gmail.com}{\nolinkurl{veith.weilnhammer@gmail.com}}\\

\newpage

\hypertarget{supplemental-information}{%
\section{Supplemental Text}\label{supplemental-information}}

\hypertarget{internal-mode-processing-is-driven-by-choice-history-as-opposed-to-stimulus-history}{%
\subsection{Internal mode processing is driven by choice history as
opposed to stimulus
history}\label{internal-mode-processing-is-driven-by-choice-history-as-opposed-to-stimulus-history}}

The main manuscript reports the effects of perceptual history, which we
defined as the impact of the choice at the preceding trial on the choice
at the current trial (henceforth \emph{choice history}). \emph{Stimulus
history}, which is defined as the impact of the stimulus presented at
the preceding trial on the choice at the present trial, represents an
alternative approach to this. Here, we compare the effects of choice
history to the effects of stimulus history.

We observed a significant bias toward stimulus history (humans: 49.76\%
± 0.1\% of trials, \(\beta\) = \(1.26\) ± \(0.94\), T(\(373.62\)) =
\(1.34\), p = \(0.18\); mice: 51.11\% ± 0.08\% of trials, T(164) = 13.4,
p = \(\ensuremath{3.86\times 10^{-28}}\)). The bias toward stimulus
history was smaller than the bias toward choice history (humans:
\(\beta\) = \(-3.53\) ± \(0.5\), T(\(66.53\)) = \(-7.01\), p =
\(\ensuremath{1.48\times 10^{-9}}\); mice: T(164) = -17.21, p =
\(\ensuremath{1.43\times 10^{-38}}\)).

The attraction of choices toward both preceding choices and stimuli is
expected, as perception was \emph{stimulus-congruent} on approximately
75\% of trials, causing choices and stimuli to be highly correlated. We
therefore compared the effects of choice history and stimulus history
after \emph{stimulus-incongruent} (i.e., \emph{error}) trials, since
those trials lead to opposite predictions regarding the perceptual
choice at the subsequent trial.

As expected from the findings presented in the main manuscript,
perceptual choices were attracted toward perceptual choices when the
inducing trial was stimulus-incongruent (i.e., a positive effect of
choice history; humans: \(\beta\) = \(0.19\) ±
\(\ensuremath{1.4\times 10^{-4}}\), z =
\(\ensuremath{1.36\times 10^{3}}\), p < \(\ensuremath{2.2\times 10^{-308}}\): mice: \(\beta\) =
\(0.92\) ± \(0.01\), z = \(88.82\), p < \(\ensuremath{2.2\times 10^{-308}}\)). By contrast, perceptual
choices tended to be repelled away from the stimulus presented at
preceding stimulus-incongruent trial (i.e., a negative effect of
stimulus history; humans: \(\beta\) = \(-0.19\) ± \(0.01\), z =
\(-16.47\), p = \(\ensuremath{5.99\times 10^{-61}}\): mice: \(\beta\) =
\(-0.92\) ± \(0.01\), z = \(-88.76\), p < \(\ensuremath{2.2\times 10^{-308}}\)). This repulsion of
choices away from stimuli presented at stimulus-incongruent trials
confirmed that choices (which are anti-correlated to stimuli at
stimulus-incongruent trials) were the primary driver of attracting
serial effects in perception.

In sum, the above results suggest that, in both humans and mice, serial
dependencies were better explained by the effects of choice history as
opposed to the effects of stimulus history. This aligns with a result
recently published for the IBL database, where mice were shown to follow
an \emph{action-kernel} as opposed to a \emph{stimulus-kernel} model
when integrating information across trials\textsuperscript{81}.

\hypertarget{fluctuations-between-internal-and-external-mode-modulate-perceptual-performance-beyond-the-effect-of-general-response-biases}{%
\subsection{Fluctuations between internal and external mode modulate
perceptual performance beyond the effect of general response
biases}\label{fluctuations-between-internal-and-external-mode-modulate-perceptual-performance-beyond-the-effect-of-general-response-biases}}

The hypothesis that perception cycles through opposing internally- and
externally-biased modes is motivated by the assumption that recurring
intervals of stronger perceptual history temporally reduce the
participants' sensitivity to external information. Importantly, the
history-dependent biases that characterize internal mode processing must
be differentiated from general response biases. In binary perceptual
decision-making, general response biases are defined by a propensity to
choose one of the two outcomes more often than the alternative. Indeed,
human participants selected the more frequent of the two possible
outcomes in 58.71\% ± 0.22\% of trials, and mice selected the more
frequent of the two possible outcomes in 54.6\% ± 0.3\% of trials.

Two caveats have to be considered to make sure that the effect of
history-congruence is distinct from the effect of general response
biases. First, history-congruent states become more likely for larger
response biases that cause an increasing imbalance in the likelihood of
the two outcomes (humans: \(\beta\) = \(0.24\) ±
\(\ensuremath{6.93\times 10^{-4}}\),
T(\(\ensuremath{2.09\times 10^{6}}\)) = \(342.43\), p < \(\ensuremath{2.2\times 10^{-308}}\); mice:
\(\beta\) = \(0.15\) ± \(\ensuremath{8.25\times 10^{-4}}\),
T(\(\ensuremath{1.32\times 10^{6}}\)) = \(181.93\), p < \(\ensuremath{2.2\times 10^{-308}}\)). One may
thus ask whether the autocorrelation of history-congruence could be
entirely driven by general response biases.

Importantly, our autocorrelation analyses account for general response
biases by computing group-level autocorrelations (Figs 2-4B) relative
to randomly permuted data (i.e., by subtracting the autocorrelation of
randomly permuted data from the raw autocorrelation curve). This
precludes that general response biases contribute to the observed
autocorrelation of history-congruence (see S5 Fig for a
visualization of the correction procedure for simulated data with
general response biases ranging from 60 to 90\%).

Second, it may be argued that fluctuations in perceptual performance may
be solely driven by ongoing changes in the strength of general response
biases. To assess the links between dynamic fluctuations in
stimulus-congruence on the one hand and history-congruence as well as
general response bias on the other hand, we computed all variables as
dynamic probabilities in sliding windows of ± 5 trials (Fig 1C).
Linear mixed effects modeling indicated that fluctuations in
history-congruent biases were larger in amplitude than the corresponding
fluctuations in general response biases in humans (\(\beta_0\) =
\(0.03\) ± \(\ensuremath{7.34\times 10^{-3}}\), T(\(64.94\)) = \(4.46\),
p = \(\ensuremath{3.28\times 10^{-5}}\)), but slightly smaller in mice
(\(\beta_0\) = \(\ensuremath{-5.26\times 10^{-3}}\) ±
\(\ensuremath{4.67\times 10^{-4}}\),
T(\(\ensuremath{2.12\times 10^{3}}\)) = \(-11.28\), p =
\(\ensuremath{1.02\times 10^{-28}}\)).

Crucially, ongoing fluctuations in history-congruence had a significant
negative effect on stimulus-congruence (humans: \(\beta_1\) = \(-0.05\)
± \(\ensuremath{5.63\times 10^{-4}}\),
T(\(\ensuremath{2.1\times 10^{6}}\)) = \(-84.21\), p < \(\ensuremath{2.2\times 10^{-308}}\); mice:
\(\beta_1\) = \(-0.12\) ± \(\ensuremath{7.17\times 10^{-4}}\),
T(\(\ensuremath{1.34\times 10^{6}}\)) = \(-168.39\), p < \(\ensuremath{2.2\times 10^{-308}}\)) beyond
the effect of ongoing changes in general response biases (humans:
\(\beta_2\) = \(-0.06\) ± \(\ensuremath{5.82\times 10^{-4}}\),
T(\(\ensuremath{2.1\times 10^{6}}\)) = \(-103.51\), p < \(\ensuremath{2.2\times 10^{-308}}\); mice:
\(\beta_2\) = \(-0.03\) ± \(\ensuremath{6.94\times 10^{-4}}\),
T(\(\ensuremath{1.34\times 10^{6}}\)) = \(-48.14\), p < \(\ensuremath{2.2\times 10^{-308}}\)). In sum,
the above control analyses confirmed that, in both humans and mice, the
observed influence of preceding choices on perceptual decision-making
cannot be reduced to general response biases.

\hypertarget{internal-mode-is-characterized-by-lower-thresholds-as-well-as-by-history-dependent-changes-in-biases-and-lapses}{%
\subsection{Internal mode is characterized by lower thresholds as well
as by history-dependent changes in biases and
lapses}\label{internal-mode-is-characterized-by-lower-thresholds-as-well-as-by-history-dependent-changes-in-biases-and-lapses}}

Random or stereotypical responses may provide an alternative explanation
for the reduced sensitivity to external sensory information that we
attribute to internal mode processing. To test this hypothesis, we asked
whether history-independent changes in biases and lapses may provide an
alternative explanation of the reduced sensitivity during internal mode.

To this end, we estimated full and history-conditioned psychometric
curves to investigate how internal and external mode relate to biases
(i.e., the horizontal position of the psychometric curve), lapses (i.e.,
the asymptotes of the psychometric curve) and thresholds (i.e.,
1/sensitivity, estimated from the slope of the psychometric curve). We
used a maximum likelihood procedure to predict trial-wise choices \(y\)
(\(y = 0\) and \(y = 1\) for outcomes A and B respectively) from the
choice probabilities \(y_p\). \(y_p\) was computed from the
difficulty-weighted inputs \(s_w\) via a parametric error function
defined by the parameters \(\gamma\) (lower lapse), \(\delta\) (upper
lapse), \(\mu\) (bias) and \(t\) (threshold; see Methods for details):

\begin{equation}
y_p = \gamma + (1 - \gamma - \delta) *  (erf(\frac{s_w + \mu}{t}) + 1) / 2
\end{equation}

Under our main hypothesis that periodic reductions in sensitivity to
external information are driven by increases in the impact of perceptual
history, one would expect (i) a history-dependent increase in biases and
lapses (effects of perceptual history), and (ii), a history-independent
increase in threshold (reduced sensitivity to external information).
Conversely, if what we identified as internal mode processing was in
fact driven by random choices, one would expect (i), a
history-independent increase in lapses (choice randomness), (ii), no
change in bias (no effect of perceptual history), and (iii), reduced
thresholds (reduced sensitivity to external information).

\hypertarget{humans}{%
\subsubsection{Humans}\label{humans}}

Across all data provided by the Confidence database\textsuperscript{20}
(i.e., irrespective of the preceding perceptual choice \(y_{t-1}\)),
biases \(\mu\) were distributed around zero (-0.05 ± 0.03; \(\beta_0\) =
\(\ensuremath{7.37\times 10^{-3}}\) ± \(0.09\), T(\(36.8\)) = \(0.08\),
p = \(0.94\); S6A-B Fig, upper panel). When conditioned
on perceptual history, biases \(\mu\) varied according to the preceding
perceptual choice, with negative biases for \(y_{t-1} = 0\) (-0.22 ±
0.04; \(\beta_0\) = \(0.56\) ± \(0.12\), T(\(43.39\)) = \(4.6\), p =
\(\ensuremath{3.64\times 10^{-5}}\); S6A-B Fig, upper
panel) and positive biases for \(y_{t-1} = 1\) (0.29 ± 0.03; \(\beta_0\)
= \(0.56\) ± \(0.12\), T(\(43.39\)) = \(4.6\), p =
\(\ensuremath{3.64\times 10^{-5}}\); S6A-B Fig, lower
panel). Absolute biases \(|\mu|\) were larger in internal mode (1.84 ±
0.03) as compared to external mode (0.86 ± 0.02; \(\beta_0\) = \(-0.62\)
± \(0.07\), T(\(45.62\)) = \(-8.38\), p =
\(\ensuremath{8.59\times 10^{-11}}\); controlling for differences in
lapses and thresholds).

Lower and upper lapses amounted to \(\gamma\) = \(0.13\) ±
\(\ensuremath{2.83\times 10^{-3}}\) and \(\delta\) = \(0.1\) ±
\(\ensuremath{2.45\times 10^{-3}}\) (S6A Fig, S6C and S6D).
Lapses were larger in internal mode (\(\gamma\) = \(0.17\) ±
\(\ensuremath{3.52\times 10^{-3}}\), \(\delta\) = \(0.14\) ±
\(\ensuremath{3.18\times 10^{-3}}\)) as compared to external mode
(\(\gamma\) = \(0.1\) ± \(\ensuremath{2.2\times 10^{-3}}\), \(\delta\) =
\(0.08\) ± \(\ensuremath{2\times 10^{-3}}\); \(\beta_0\) = \(-0.05\) ±
\(\ensuremath{5.73\times 10^{-3}}\), T(\(47.03\)) = \(-9.11\), p =
\(\ensuremath{5.94\times 10^{-12}}\); controlling for differences in
biases and thresholds).

Conditioning on the previous perceptual choice revealed that the
between-mode difference in lapse was not general, but depended on
perceptual history: For \(y_{t-1} = 0\), only higher lapses \(\delta\)
differed between internal and external mode (\(\beta_0\) = \(-0.1\) ±
\(\ensuremath{9.58\times 10^{-3}}\), T(\(36.87\)) = \(-10.16\), p =
\(\ensuremath{3.06\times 10^{-12}}\)), whereas lower lapses \(\gamma\)
did not (\(\beta_0\) = \(0.01\) ± \(\ensuremath{7.77\times 10^{-3}}\),
T(\(33.1\)) = \(1.61\), p = \(0.12\)). Vice versa, for \(y_{t-1} = 1\),
lower lapses \(\gamma\) differed between internal and external mode
(\(\beta_0\) = \(-0.11\) ± \(0.01\), T(\(40.11\)) = \(-9.59\), p =
\(\ensuremath{6.14\times 10^{-12}}\)), whereas higher lapses \(\delta\)
did not (\(\beta_0\) = \(0.01\) ± \(\ensuremath{7.74\times 10^{-3}}\),
T(\(33.66\)) = \(1.58\), p = \(0.12\)).

Thresholds \(t\) were estimated at 3 ± 0.06 (S6A Fig and
S6E). Thresholds \(t\) were larger in internal mode (3.66 ± 0.09) as
compared to external mode (2.02 ± 0.03; \(\beta_0\) = \(-1.77\) ±
\(0.25\), T(\(50.45\)) = \(-7.14\), p =
\(\ensuremath{3.48\times 10^{-9}}\); controlling for differences in
biases and lapses). In contrast to the bias \(\mu\) and the lapse rates
\(\gamma\) and \(\delta\), thresholds \(t\) were not modulated by
perceptual history (\(\beta_0\) = \(0.04\) ± \(0.06\),
T(\(\ensuremath{2.97\times 10^{3}}\)) = \(0.73\), p = \(0.47\)).

\hypertarget{mice}{%
\subsubsection{Mice}\label{mice}}

When estimated based on the full dataset provided in the IBL
database\textsuperscript{21} (i.e., irrespective of the preceding
perceptual choice \(y_{t-1}\)), biases \(\mu\) were distributed around
zero (\(\ensuremath{3.87\times 10^{-3}}\) ±
\(\ensuremath{9.81\times 10^{-3}}\); T(164) = 0.39, p = \(0.69\);
S7A-B Fig, upper panel). When conditioned on the
preceding perceptual choice, biases were negative for \(y_{t-1} = 0\)
(\(-0.02\) ± \(\ensuremath{8.7\times 10^{-3}}\); T(\(164) = -1.99\), p =
\(0.05\); S7A-B Fig, middle panel) and positive for
\(y_{t-1} = 1\) (\(0.02\) ± \(\ensuremath{9.63\times 10^{-3}}\);
T(\(164\)) = \(1.91\), p = \(0.06\); S7A-B Fig, lower
panel). As in humans, mice showed larger biases during internal mode
(\(0.14\) ± \(\ensuremath{7.96\times 10^{-3}}\)) as compared to external
mode (\(0.07\) ± \(\ensuremath{8.7\times 10^{-3}}\); \(\beta_0\) =
\(-0.18\) ± \(0.03\), T = \(-6.38\), p =
\(\ensuremath{1.77\times 10^{-9}}\); controlling for differences in
lapses and thresholds).

Lower and upper lapses amounted to \(\gamma\) = \(0.1\) ±
\(\ensuremath{4.35\times 10^{-3}}\) and \(\delta\) = \(0.11\) ±
\(\ensuremath{4.65\times 10^{-3}}\) (S7A Fig, S7C, and S7D).
Lapse rates were higher in internal mode (\(\gamma\) = \(0.15\) ±
\(\ensuremath{5.14\times 10^{-3}}\), \(\delta\) = \(0.16\) ±
\(\ensuremath{5.79\times 10^{-3}}\)) as compared to external mode
(\(\gamma\) = \(0.06\) ± \(\ensuremath{3.11\times 10^{-3}}\), \(\delta\)
= \(0.07\) ± \(\ensuremath{3.34\times 10^{-3}}\); \(\beta_0\) =
\(-0.11\) ± \(\ensuremath{4.39\times 10^{-3}}\), T = \(-24.8\), p =
\(\ensuremath{4.91\times 10^{-57}}\); controlling for differences in
biases and thresholds).

For \(y_{t-1} = 0\), the difference between internal and external mode
was more pronounced for higher lapses \(\delta\) (T(164) = \(21.44\), p
= \(\ensuremath{1.93\times 10^{-49}}\)). Conversely, for
\(y_{t-1} = 1\), the difference between internal and external mode was
more pronounced for lower lapses \(\gamma\) (T(\(164\)) = \(-18.24\), p
= \(\ensuremath{2.68\times 10^{-41}}\)). In contrast to the human data,
higher lapses \(\delta\) and lower lapses \(\gamma\) were significantly
elevated during internal mode irrespective of the preceding perceptual
choice (higher lapses \(\delta\) for \(y_{t-1} = 1\): T(164) = -2.65, p
= \(\ensuremath{8.91\times 10^{-3}}\); higher lapses \(\delta\) for
\(y_{t-1} = 0\): T(164) = -28.29, p =
\(\ensuremath{5.62\times 10^{-65}}\); lower lapses \(\gamma\) for
\(y_{t-1} = 1\): T(164) = -32.44, p =
\(\ensuremath{2.92\times 10^{-73}}\); lower lapses \(\gamma\) for
\(y_{t-1} = 0\): T(164) = -2.5, p = \(0.01\)).

In mice, thresholds \(t\) amounted to \(0.15\) ±
\(\ensuremath{6.52\times 10^{-3}}\) (S7A Fig and S7E) and
were higher in internal mode (\(0.27\) ± \(0.01\)) as compared to
external mode (\(0.09\) ± \(\ensuremath{4.44\times 10^{-3}}\);
\(\beta_0\) = \(-0.28\) ± \(0.04\), T = \(-7.26\), p =
\(\ensuremath{1.53\times 10^{-11}}\); controlling for differences in
biases and lapses). Thresholds \(t\) were not modulated by perceptual
history (T(164) = 0.94, p = \(0.35\)).

In sum, the above analyses showed that, in both humans and mice,
internal and external mode differ with respect to biases, lapses and
thresholds. Internally-biased processing was characterized by higher
thresholds, indicating a reduced sensitivity to sensory information, as
well as by larger biases and lapses. Importantly, between-mode
differences in biases and lapses strongly depended on perceptual
history. This confirmed that internal mode processing cannot be
explained solely on the ground of a general (i.e., history-independent)
increase in lapses or bias indicative of random of stereotypical
responses.

\hypertarget{internal-mode-processing-can-not-be-reduced-to-insufficient-task-familiarity}{%
\subsection{Internal mode processing can not be reduced to insufficient
task
familiarity}\label{internal-mode-processing-can-not-be-reduced-to-insufficient-task-familiarity}}

It may be assumed that participants tend to repeat preceding choices
when they are not yet familiar with the experimental task, leading to
history-congruent choices that are caused by insufficient training. To
assess this alternative explanation, we contrasted the correlates of
bimodal inference with training effects in humans and mice.

\hypertarget{humans-1}{%
\subsubsection{Humans}\label{humans-1}}

In the Confidence database\textsuperscript{20}, training effects were
visible from RTs that were shortened by increasing exposure to the task
(\(\beta\) = \(\ensuremath{-7.53\times 10^{-5}}\) ±
\(\ensuremath{6.32\times 10^{-7}}\),
T(\(\ensuremath{1.81\times 10^{6}}\)) = \(-119.15\), p < \(\ensuremath{2.2\times 10^{-308}}\)).
Intriguingly, however, history-congruent choices became more frequent
with increased exposure to the task (\(\beta\) =
\(\ensuremath{3.6\times 10^{-5}}\) ±
\(\ensuremath{2.54\times 10^{-6}}\), z = \(14.19\), p =
\(\ensuremath{10^{-45}}\)), speaking against the proposition that
insufficient training induces seriality in response behavior.

\hypertarget{mice-1}{%
\subsubsection{Mice}\label{mice-1}}

As in humans, it is an important caveat to consider whether the observed
serial dependencies in mice reflect a phenomenon of perceptual
inference, or, alternatively, an unspecific strategy that occurs at the
level of reporting behavior. We reasoned that, if mice indeed tended to
repeat previous choices as a general response pattern, history effects
should decrease during training of the perceptual task. We therefore
analyzed how stimulus- and history-congruent perceptual choices evolved
across sessions in mice that, by the end of training, achieved
proficiency (i.e., stimulus-congruence \(\geq\) 80\%) in the
\emph{basic} task of the IBL dataset\textsuperscript{21}.

Across sessions, we found that stimulus-congruent perceptual choices
became more frequent (\(\beta\) = \(0.34\) ±
\(\ensuremath{7.13\times 10^{-3}}\),
T(\(\ensuremath{8.51\times 10^{3}}\)) = \(47.66\), p < \(\ensuremath{2.2\times 10^{-308}}\)) and TDs
were progressively shortened (\(\beta\) = \(-22.14\) ± \(17.06\),
T(\(\ensuremath{1.14\times 10^{3}}\)) = \(-1.3\), p < \(\ensuremath{2.2\times 10^{-308}}\)). Crucially,
the frequency of history-congruent perceptual choices also increased
during training (\(\beta\) = \(0.13\) ±
\(\ensuremath{4.67\times 10^{-3}}\),
T(\(\ensuremath{8.4\times 10^{3}}\)) = \(27.04\), p =
\(\ensuremath{1.96\times 10^{-154}}\); S8 Fig).\\
Within individual session, longer task exposure was associated with an
increase in history-congruence (\(\beta\) =
\(\ensuremath{3.6\times 10^{-5}}\) ±
\(\ensuremath{2.54\times 10^{-6}}\), z = \(14.19\), p =
\(\ensuremath{10^{-45}}\)) and a decrease in TDs (\(\beta\) = \(-0.1\) ±
\(\ensuremath{3.96\times 10^{-3}}\),
T(\(\ensuremath{1.34\times 10^{6}}\)) = \(-24.99\), p =
\(\ensuremath{9.45\times 10^{-138}}\)). In sum, these findings strongly
argue against the proposition that mice show biases toward perceptual
history due to an unspecific response strategy.

\end{document}
